\section{Mängdlära}
% Lägg till komplement

\begin{definition}[Mängd]
En mängd är en samling element där ordning och antal inte spelar någon roll. Dessa element kan vara vad som helst till exempel färgen blå eller talet 2
\end{definition}

Man kan definiera en mängd antingen genom att lista upp dess element inom måsvingar, till exempel \(\left\{\text{flygplan, hatt, pannkaka} \right\}\), eller genom en regel för elementen, till exempel mängden av alla färger.

\begin{definition}[Mängdoperationer]
	En mängdoperator tar en eller flera mängder och skapar en ny. Några av de vanligaste är
	\begin{center}
		\begin{tabular}{|ccc|}
			\hline
			Operation & Symbol & Innehåller alla element som\\ \hline
			Union & \(A\cup B\) & Är i A eller B \\ \hline
			Snitt & \(A \cap  B\) & Är i A och B \\ \hline
			Komplement & \(A^C\) & Inte är i A \\ \hline
		\end{tabular}
	\end{center}
	Notera! Komplement fungerar bara om det finns universalmängd som berättar vad alla relevanta element är, den är oftast intuitiv.
\end{definition}

\begin{problem}
	Är \(\left\{1, 2, 3\right\}\) samma som \(\left\{3, 1, 1, 2\right\}\)?
\end{problem}

\begin{problem}
	Är \(\left\{\text{blå} , \text{röd} \right\}\) samma som \(\left\{\text{blå}, \left\{\text{röd} \right\}\right\}\)?
\end{problem}

\begin{problem}
	Skriv upp mängden av alla heltal \(x\) som uppfyler påståendet ``\(x>2\) och \(x <5\)''.
\end{problem}


\begin{problem}
	Om universalmängden är \(\left\{\text{Mattekollo, UVS och Kodsport} \right\}\), vad blir då:
	\begin{enumerate}
		\item \(\left\{\text{Mattekollo, Kodsport} \right\} \cup \left\{\text{Kodsport} \right\}\) 
		\item \(\left\{\text{UVS, Mattekollo}\right\}^C\) 
		\item \(\left\{\text{Mattekollo, UVS}\right\} \cap \left\{\text{UVS, Kodsport} \right\} \cap \left\{\text{Kodsport, Mattekollo} \right\}\) 
	\end{enumerate}
	
	OBS! UVS = Ung Vetenskapssport.
\end{problem}

\begin{problem}
	Vi har mängden av alla heltal som är delbara med 7. Vad blir dess komplement? (Universalmängden är alla heltal)
\end{problem}

\begin{problem}
	Du har mängden av alla tidigare mattekollo deltagare \(X\)  och mängden av alla mattekollo ledare \(Y\) . Hur skapar man mängden av alla mattekollo ledare som tidigare varit deltagare?
\end{problem}

\begin{problem}
	Vi har mängden A vars element uppfyller påståendet ``Elementet är rött'' och B vars element uppfyller påståendet ``Elementet är ett klädesplagg''. Vilket påstående uppfyller deras snitt?
\end{problem}

\begin{problem}
	Finns det en koppling mellan mängdoperationerna och påståendet deras element uppfyller? Dessa beror kanske särskilt på de logiska konjukntionerna?
\end{problem}

\begin{problem}
	Stämmer det att \((A \cup B) \cap (A \cup C) = A \cup (B \cap C)\) för alla mängder \(A, B, C\)? Använd en sanningstabell!
\end{problem}

\begin{problem}[Extra]
	Det är 9 dagar på Mattekollo, 7 dagar har lektioner, 2 är måndagar, och en är varken måndag eller har lektioner. Hur många måndagar har lektioner?
	Hur många element uppfyller påståendet?
\end{problem}

\begin{problem}[Extra]
	Hur många element är i \((A \cup B) \cap C\) om \(A =\) mängden av alla tal delbara på 2, \(B = \) mängden av alla tal delbara med 5 och \(C =\) mängden av alla tal från 0 till 100.
\end{problem}

\begin{problem}
	Ett Venndiagram används illustrerar hur mängder hör ihop. Varje mängd ritas som en cirkel så att de överlappar i om mängderna har något element gemensamtom mängderna har något element gemensamt. Sedan färgar man i den mängd man menar. Till blir venndiagrammet för mängden \(A \cup B\)
	\def\firstcircle{(0,0) circle (1.5cm)}
	\def\secondcircle{(60:2cm) circle (1.5cm)}
	\def\thirdcircle{(0:2cm) circle (1.5cm)}
	\begin{center}
		\begin{tikzpicture}
			\begin{scope}
%				\fill[red, fill opacity=0.3] \firstcircle;
%				\fill[green, fill opacity=0.3] \secondcircle;
%				\fill[blue, fill opacity=0.3] \thirdcircle;
				\draw \firstcircle node[below] {$A$};
				\draw \secondcircle node [above] {$B$};
				\draw \thirdcircle node [below] {$C$};
				\fill[opacity=0.5, kollofargen] \firstcircle \secondcircle;
			\end{scope}
		\end{tikzpicture}
	\end{center}

	Rita ett venndiagram för \((A \cap B)\cup C\). Kan man visa uppgift 9 med dessa?
\end{problem}

