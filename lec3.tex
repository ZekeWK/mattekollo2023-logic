\section{Mängdlära}
% Lägg till komplement

\begin{definition}[Mängd]
En mängd är en samling ``saker'' där ordning och antal inte spelar någon roll. Dessa "saker" som mängden innehåller kallas element. En mängd noteras normalt antingen med alla sina element inom måsvingar, till exempel \(\left\{1, -2, 10, \frac{1}{7}, \sqrt{2}  \right\}\) eller \(\left\{\text{blå, hatt, pannkakor} \right\}\), eller som en regel, till exempel mängden av alla färger.
\end{definition}

\begin{problem}
	Är \(\left\{1, 2, 3\right\}\) samma som \(\left\{3, 1, 1, 2\right\}\)?
\end{problem}

\begin{problem}
	Är \(\left\{\sqrt{2}, 1 \right\}\) samma som \(\left\{\sqrt{2} , \left\{1\right\}\right\}\)?
\end{problem}

\begin{problem}
	Vad är mängden av alla element \(x\) som uppfyler påståendet ``x>2 \land x <5''?
\end{problem}

\begin{definition}[Element]
	Om \(A\) är en mängd och \(a\) ett element i \(A\) skriver man att \(a \in A\) där \(\in\) översätts till ``element i''.
\end{definition}

\begin{problem}
	Stämmer det att:
	\begin{itemize}
		\item \(2 \in \left\{1, 3, 5\right\}\)
		\item 2 \(\in\) mängden av alla jämna tal 
	\end{itemize}
\end{problem}

\begin{definition}[Mängdoperationer]
	En mängdoperator tar en eller flera mängder och skapar en ny. Några av de vanligaste är
	\begin{center}
		\begin{tabular}{|ccc|}
			\hline
			Operation & Symbol & Innehåller alla element som\\ \hline
			Union & \(A\cup B\) & Är i A eller B \\ \hline
			Snitt & \(A \cap  B\) & Är i A och B \\ \hline
			Differens & \(A \setminus  B\) & Är i A och inte är i B \\ \hline
			Komplement & \(A^C\) & Inte är i A \\ \hline
		\end{tabular}
	\end{center}
	Notera! Komplement fungerar bara om det finns en tydlig mängd som alla är del av.
\end{definition}

\begin{problem}
	Alla påståenden i im
\end{problem}

\begin{problem}
	Stämmer dessa påståenden?
\end{problem}

\begin{problem}
	Vad är mängden av alla objekt ur A som uppfyller påstendet:
\end{problem}

\begin{problem}
	Mängden A innehåller alla personer som uppfyller påstående a, samma för B och b, vilken mängd innehåller objekten som uppfyller:
	\begin{itemize}
		\item A och B
		\item A eller B
		\item Antingen A eller B
	\end{itemize}
\end{problem}

\begin{problem}
	Vad blir mängdoperationerna som uppfyller:
\end{problem}

\begin{problem}
	Hur många element uppfyller påståendet?
\end{problem}

\begin{problem}[Extra]
	Typiska mängdproblem
\end{problem}








\begin{itemize}
	\item Vad är en mängd + exempel
	\item Vad kan vi göra för operationer på mängder?
	\item Påståenden om element
	\item Hur konjunktioner agerar på mängden av element som möter påståendet
	\item Extrauppgifter: Beräkna antalet element i dessa mängder. Skapa dessa mängder, Venn diagram
\end{itemize}
