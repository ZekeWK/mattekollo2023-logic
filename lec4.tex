\begin{definition}[Krets]
	En logisk krets är ett annat sätt skriva påståenden. Man låter några påståenden representeras med variabler, och använder sedan konjunktioner och negationer för att skapa ett nytt påstående. Noderna är

	\begin{circuitikz} \draw
		(0,0) node[not port] (not) {}
		(0,-1) node[below] {``inte''}
		(4,0) node[and port] (and) {}
		(3.5,-1) node[below] {``och''}
		(8,0) node[or port]  (or)  {}
		(7.5,-1) node[below] {``eller''}
		(12,0) node[xor port]  (xor)  {}
		(11.5,-1) node[below] {``antingen eller''};
	\end{circuitikz}
	
	\noindent
	De kopplas ihop genom att dra en linje från det högra utvärdet på en nod till det vänstra invärdet.
\end{definition}

\begin{problem}
	Vilka av påståendena \(A, B, C\) ska vara sanna för \(y\) ska vara sann?

	\begin{circuitikz} \draw
		(0,4) node (A) {\(A\)}
		(0,2) node (B) {\(B\)}
		(0,0) node (C) {\(C\)}
		(3,3) node[and port] (and) {}
		(3,1) node[xor port] (xor) {}
		(6,2) node[or port] (or) {}
		(8,2) node (y) {\(y\)}
		(A) -- (and.in 1)
		(B) -- (and.in 2)
		(B) -- (xor.in 1)
		(C) -- (xor.in 2)
		(and.out) -- (or.in 1)
		(xor.out) -- (or.in 2)
		(or.out) -- (y);
	\end{circuitikz}
\end{problem}

\begin{problem}
	Rita upp en egen krets för påståendet ``Antingen \(A\) och \(B\) eller \(C\)''.
\end{problem}

\begin{problem}
	I en masugn måste man bestämma huruvida värmen ska vara på eller av. I en förenklad model säger vi att värmen ska vara på om temperaturen är under 1000\textdegree~eller om den är inställd på ``maximal effekt''. Så klart måste värmen vara avstängd om ''Nödstopp`` knappen är aktiverad. Skapa en logisk krets för detta.
\end{problem}


\section*{Att bygga en miniräknare}
\begin{problem}
	Ställ upp additionerna med binära tal:
	\begin{itemize}
		\item \(1_2 + 11_2\) 
		\item \(101_2 + 1010_2\) 
		\item \(111_2 + 1_2\) 
	\end{itemize}
\end{problem}


\begin{problem}
	Vi vill beräkna summan av två ensiffriga binära tal. Vi har påståendena ``Ena talet är \(1_2\)'' och ``Andra talet är \(1_2\)'' (istället för \(0_2\)). Hur kan man skriva påståendena ``Andra siffran i summan är \(1_2\)'' respektive ``Första siffran i summan är \(1_2\)''. Rita en krets för dessa!
\end{problem}

\begin{problem}
	Nu vill vi bestämma en godtycklig siffra i det binära talet. Vi vet siffrorna på samma position för talet och har påstående för att dessa är ettor. Vi har också ett påstående för om det finns en ``minnes etta'' från siffran innan. Rita en krets för om siffran och för om den leder till en minnes etta.
\end{problem}

\begin{problem}
	Skapa en symbol för din tidigare krets, en förkortning där bara invärd och utvärde syns men logiken blir samma som tidigare. Sätt ihop dessa till att kunna beräkna addition av två stycken fyra bitars tal.
\end{problem}

\begin{problem}[Extra]
	Positiva tal är bra, men det är trevligt med negativa tal också. Om vi bara har 4 siffror räknar vi i modulo 16. Detta gör att \(15 \equiv -1 \mod 16, \dots , 8 \equiv -8\). Blir \(3 + 15 \equiv 3 - 1 \mod 16\)? Hur kan vi skriva negativa tal i vår dator? Fungerar det med vår krets?
\end{problem}

\begin{problem}[Extra]
	Logiska kretsar kan inte spara någon information, men i datorer vill vi kunna göra detta. Låt oss utvärdera samma krets flera gånger, men låta ett av utvärdena påverka invärdet nästa gång. Vi har alltså ett par av invärde utvärde så att invärdet är samma som utvärdet var gången innan. För att göra det tydligare hur de hänger ihop kan vi skriva dem som en kvadrat med in och utvärde, men detta är bara för att förenkla. Vilka typer av kretsar kan vi skapa utifrån detta?
\end{problem}


% \section{Att bygga ett RAM}
% Nu kommer vi gå aningen bortom vanlig logik för att kunna bättre beskriva datorer. Datorer gör inte allt i logisk krets en gång, utom de använder minne som inputs i sina kretsar som de även kan ändra på. Låt oss göra något liknande.
% 
% Vi lägger till en extra input
