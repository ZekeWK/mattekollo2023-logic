\begin{definition}[Slutledning]
	En slutledning innebär att man vet att ett påstående, konsekvensen, är sant om ett annat påstående, villkoret, är sant. Slutledningen i sig är falsk bara om villkoret är sant men inte konsekvensen.
\end{definition}

\begin{definition}[Sanningstabell] % Lägg till om ekvivalens mellan kretsar
	En sanningstabell berättar exakt när ett påstående är sant. Varje rad motsvarar ett fall och varje kolumn ett påstående. De kan till exempel se ut såhär:
	\begin{center}
		\begin{tabular}{|c|c|c|}
			\hline
			``Det regnar'' & ``William är inne'' & ``William är inne om det regnar'' \\ \hline
			F & F & S \\ \hline
			F & S & S \\ \hline
			S & F & F \\ \hline
			S & S & S \\ \hline
		\end{tabular}
	\end{center}
\end{definition}

\begin{problem}
	Varför är slutledningen ``Det är sommarlov om det är mattekollo'' sann även under sportlovet?
\end{problem}

\begin{problem}
	Är slutledningarna sanna?
	\begin{itemize}
		\item Eftersom Julia är en ledare på mattekollo och Julia tycker om astronomi finns det ledare på mattekollo som gillar astronomi.
		\item Eftersom Harry gillar schack och schack är ett brädspel gillar Harry alla brädspel.
	\end{itemize}
\end{problem}

\begin{problem}
	Vad är sanningstabellen för ``Linköpings universitet är öppet eller TODO''
\end{problem}

\begin{problem}
	Vilket påstående har sanningstabellen
	\begin{center}
		\begin{tabular}{|c|c|c|}
			\hline
			``Påståendet är sant'' & ``Påståendet är falskt'' & ~~~~~~~~???~~~~~~~~ \\ \hline
			F & F & F \\ \hline
			F & S & S \\ \hline
			S & F & S \\ \hline
			S & S & F \\ \hline
		\end{tabular}
	\end{center}
\end{problem}

\noindent
\textbf{Notera!}  Människor talar alltid sanning och varulvar ljuger alltid!

\begin{problem}
	Efter en fullmåne under mattekollo inser du att några av ledarna är varulvar! Du träffar Lisa och Sebastian. Sebastian påstår att de båda är människor, men Lisa påstår att Sebastian är en varulv? Vem är vad, och varför?
\end{problem}

\begin{problem}
	Du springer vidare och träffar Erik och Tobias. Erik säger att man stoppar ledare från att vara varulvar genom att ge dem kaffe. Tobias påstår att Erik är en varulv om han själv är det.

	Skapa en sanningstabell med påståendena ``Erik är människa'' ``Tobias är människa'' ``Erik talar sanning'' ``Tobias talar sanning'' ``Människor talar alltid sanning, varulvar ljuger alltid''. Kan denna berätta vem som är vad? Är kaffe räddningen?
\end{problem}

\begin{problem}
	På väg in i köket möter du Harry, Elias, och William. Du tycker dig höra Harry säga ``Elias är varulv'', Elias säga ``William är varulv'' och William säga ``Harry är varulv''. Kan du ha hört rätt?
\end{problem}

\begin{problem}[Extra]
	TODO
	En är en varulv och en är en människa, men du vet inte vem som är vad. Du får ställa en fråga för att lista ut vilket kaffepaket du ska ta? Vilken fråga?
\end{problem}

\begin{problem}
	Är påståendet ``Benjamin är varulv och Valentina är Varulv'' negationen till ``Benjamin är människa eller Valentina är människa''? Två påståenden är samma om de har samma sanningstabell.
\end{problem}

\begin{problem}
	Innan du hinner sätta igång kaffemaskinen kommer alla mattekollo ledare fram. Fredrik kommer fram och säger ``Alla mattekollo ledare dricker kaffe'' och Kevin säger att ``Någon mattekollo ledare dricker inte kaffe''. Sedan TODO

	Följande problem kan lösas med en stor sanningstabell, men kan man göra den lite mindre?
\end{problem}

\begin{problem}[Extra]
	När dagen är räddad och alla ledare återigen är lika snälla som de brukar. 
	En sats, kan vi bevisa den? Vilka är de tydliga stegen och implikationerna.
	De ber dig bevisa att TODO
\end{problem}

