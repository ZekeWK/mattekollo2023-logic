\begin{definition}[Slutledning]
	En slutledning innebär att man vet att ett påstående, konsekvensen, är sant om ett annat påstående, villkoret, är sant. Slutledningen i sig är falsk bara om villkoret är sant men inte konsekvensen.
\end{definition}

\begin{definition}[Sanningstabell] % Lägg till om ekvivalens mellan kretsar
	En sanningstabell berättar exakt när ett påstående är sant. Varje rad motsvarar ett fall och varje kolumn ett påstående. De kan till exempel se ut såhär:
	\begin{center}
		\begin{tabular}{|c|c|c|}
			\hline
			``Det regnar'' & ``William är inne'' & ``William är inne om det regnar'' \\ \hline
			F & F & S \\ \hline
			F & S & S \\ \hline
			S & F & F \\ \hline
			S & S & S \\ \hline
		\end{tabular}
	\end{center}
\end{definition}

\begin{problem}
	Varför är slutledningen ``Det är sommarlov om det är Mattekollo'' sann även under sportlovet?
\end{problem}

\begin{problem}
	Är slutledningarna garanterat sanna?
	\begin{enumerate}[label=\alph*)]
		\item Eftersom Julia är en ledare på Mattekollo och Julia tycker om astronomi, finns det en ledare på Mattekollo som gillar astronomi.
		\item Eftersom Harry gillar Mao och Mao är ett kortspel, gillar Harry alla kortspel.
	\end{enumerate}
\end{problem}

\begin{problem}
	Vad är sanningstabellen för ``Om det är lektioner eller kvällsaktiviteter så är jag med''?
\end{problem}

\begin{problem}% Byt till sanningstabellen
	Låt \(X\) och \(Y\) vara påståenden. Vad blir sanningstabellen för ``\(X\) om \(Y\)''? Vad blir sanningstabellen för ``inte \(X\) om inte \(Y\)''.
%	\begin{center}
%		\begin{tabular}{|c|c|c|}
%			\hline
%			``A'' & ``B'' &  \\ \hline
%			F & F & F \\ \hline
%			F & S & S \\ \hline
%			S & F & S \\ \hline
%			S & S & F \\ \hline
%		\end{tabular}
%	\end{center}
\end{problem}

\noindent
\textbf{Notera!}  Människor talar alltid sanning och varulvar ljuger alltid!

\begin{problem}
	Efter en fullmåne under Mattekollo inser du att några av ledarna är varulvar! Du träffar Lisa och Sebastian. Sebastian påstår att de båda är människor, men Lisa påstår att Sebastian är en varulv? Vem är vad, och varför?
\end{problem}

\begin{problem}
	Du springer vidare och träffar Erik och Tobias. Erik säger att man stoppar ledare från att vara varulvar genom att ge dem kaffe. Tobias påstår att Erik är en varulv om och endast om han själv är det.

	Skapa en sanningstabell med påståendena ``Erik är människa'' ``Tobias är människa'' ``Erik talar sanning'' ``Tobias talar sanning'' ``Människor talar alltid sanning, varulvar ljuger alltid''. Kan denna berätta vem som är vad? Är kaffe räddningen?
\end{problem}

\begin{problem}
	På väg in i köket möter du Harry, Elias, och William. Du tycker dig höra Harry säga ``Elias är varulv'', Elias säga ``William är varulv'' och William säga ``Harry är varulv''. Kan du ha hört rätt?
\end{problem}

\begin{problem}[Extra]
	Du stöter på Julia men vet inte om hon är varulv eller människa. Hon håller två kaffepaket, ett smakar gott och ett illa, men du vet inte vilket som är vilket. Du får ställa en fråga för att lista ut vilket kaffepaket du ska ta. Vilken fråga?
\end{problem}

\begin{problem}
	Är påståendet ``Benjamin är varulv och Valentina är varulv'' negationen till ``Benjamin är människa eller Valentina är människa''? Två påståenden är samma om de har samma sanningstabell och negationer om de har motsatta.
\end{problem}

\begin{problem}
	Innan du hinner sätta igång kaffemaskinen kommer de sista Mattekolloledarna fram. Fredrik säger ``Alla Mattekolloledare dricker kaffe'' medan Kevin säger att ``William dricker inte kaffe''. Kan båda vara varulvar? Är det rimligt att ställa upp en sanningstabell där man undersöker varje ledares kaffe vanor individuellt?
\end{problem}

\begin{problem}[Extra Extra]
	Några månader senare möter du tre allvetande matematikprofessorer, Sann, Falsk och Slump. Sann talar alltid sanning, Falsk ljuger alltid och vad Slump säger är helt slumpmässigt. Ditt mål är att bestämma vem som är vem genom att ställa ja och nej frågor. Varje fråga måste ges till en specifik proffessor. De förstår svenska men svarar på sitt eget språk där orden för ja och nej är ``Bouba'' och ``Kiki'', men det är oklart vilken som är vilken. Hur bestämmer du vem som är vem givet 100 frågor? Hur gör du med bara 3 frågor?
\end{problem}

