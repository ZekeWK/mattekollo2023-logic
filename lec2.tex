\section{Slutledningar, Sanningstabeller och Pussel}
% Gör historien här mer tydlig!

\begin{definition}[Slutledning]
	
\end{definition}
\begin{problem}
	Är följande slutledningar sanna?
	\begin{itemize}
		\item Eftersom Julia är en ledare på mattekollo och Julia tycker om astronomi existerar det en ledare på mattekollo som gillar astronomi.
		\item Eftersom schack är ett brädspel och Harry gillar vissa brädspel tycker Harry om schack.
	\end{itemize}
\end{problem}

\begin{problem}
	Varför är följande slutledningar falska?
	\begin{itemize}
		\item 
	\end{itemize}
	
\end{problem}


Notera! Människor talar alltid sanning och valuvar ljuger alltid! Både människor och varulvar är personer!
\begin{problem}
	Du kommer in i en by och två personer, Alice och Bob, pratar med dig. Alice säger att exakt en av dem är en varulv och Bob säger att båda är människor. Vad är dem?
\end{problem}

\begin{problem}
	Du fortsätter att bekanta dig med byborna och stöter på, Alex och Britta. Du undrar vilken väg som är bäst till bergen, sjövägen eller skogsvägen. Britta säger snabbt att sjövägen är bäst. Men då påstår Alex att exakt en av dem är en varulv. Vilken väg är bäst?
\end{problem}

%\begin{problem}
%	Efter en timmes resa stöter du på tre personer, Adam, Bill, Cecilia och Daniela. Bill påstår att Adam är varulv, Cecilia påstår att Bob inte är människa, Daniela påstår att Cecilia inte är människa och Adam påstår att Daniela är varulv. Hur många människor finns bland dem?
%\end{problem}

\begin{definition}[Sanningstabell] % Lägg till om ekvivalens mellan kretsar
	En sanningstabell är ett sätt att lättare analysera och undersöka logiska påståenden som beror av sanningsvärdet på andra påståenden. Man skriver då en rad för varje möjlig kombination av sant och falskt för dessa värden, och vad påståendet (eller påståendena) skulle ha för sanningsvärde. Exempelvis kan sanningstabellen för \((A \land B) \lor (\lnot A \land \lnot B)\) se ut så här:
	\begin{center}
		\begin{tabular}{|c|c|c|c|c|c|}
			\hline
			\(A\) & \(B\)  & \(A \land B\)  & \(\lnot A \land \lnot B\) & \((A \land B) \lor (\lnot A \land \lnot B)\)\\ \hline
			0 & 0 & 0 & 1 & 1 \\ \hline
			0 & 1 & 0 & 0 & 0 \\ \hline
			1 & 0 & 0 & 0 & 0 \\ \hline
			1 & 1 & 1 & 0 & 1 \\ \hline
		\end{tabular}
	\end{center}
\end{definition}

% Flytta bak sanningstabellen till lite senare i samma.

\begin{problem}
	Vad blir sanningstabellen för \(A \underline{\lor} B\)?
\end{problem}

\begin{problem}
	Skriv en sanningstabell över följande påståenden, bestäm själv vilka 3 variabler som blir A, B, C:
	\begin{itemize}
		\item 
	\end{itemize}
	
\end{problem}

\begin{problem}
	När berget syns och skogen tunnar ut syns Alva och Bror. Bror påstår att Alva är olik honom medan Alva påstår att Bror är en människa. Låt \(A\) vara påståendet ``Alva är en människa'' och \(B\) vara påståendet ``Bror är en människa''. Påståendet ``Puzzlet stämmer'' kan då skrivas som:
	\[
		(A \underline{\lor} B \Leftrightarrow  B) \land (B \Leftrightarrow A).
	\]
	Varför då? Kan man använda en sanningstabell på detta påstående? Vad stämmer?
\end{problem}


\begin{problem}
	En varulv fdfasf dafd
	Lös genom att själv bestämma, och beskriv som ett påstående. Kan det lösas med en sanningstabell?
\end{problem}

\begin{problem}
	Följande problem kan lösas med en stor sanningstabell, men kan man göra den lite mindre?
\end{problem}

\begin{problem}
	En sats, kan vi bevisa den? Vilka är de tydliga stegen och implikationerna.
\end{problem}

\begin{problem}
	Kan man använda föregående (sanna) slutledningar på dessa problem?
\end{problem}

\begin{problem}
	Lite olika kluringar
\end{problem}




\begin{itemize}
	\item Vi gör en tabell över vad som gäller i alla fall
	\item Göra några enklare pussel själv
	\item Kan vi göra ett matematiskt påstående av dem, då kan vi göra sanningstabeller!
	\item Kan vi göra vissa enklare, genom att använda att och och eller inte alltid förändras?
	\item Kopplar även till matematiska påståenden, visa den enklaste saken
	\item Fokusera hela tiden på att strukturerat bevisa kluringar
	\item Extra uppgifter: Några svårare pussel + Bevis
\end{itemize}
