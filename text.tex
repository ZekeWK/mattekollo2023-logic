\section{Påståenden och förhållanden}
\begin{itemize}
	\item Vad är ett påstående?
	\item Exempel på påståenden.
	\item Formalisering av negation, och, eller, exklusivt eller, alla, exisisterar, i små steg med mycket exempel.
	\item Jobba med påstående, blanda mellan att skapa egna, att svara på om de är sanna, och att ha matematiska satser, till skillnad från definitioner.
	\item Att konvertera mellan likadana påståenden.
	\item Extrauppgifter: Skapa tabeller över alla möjligheter
\end{itemize}

\section{Sanningstabeller och Pussel}
\begin{itemize}
	\item Vi gör en tabell över vad som gäller i alla fall
	\item Göra några enklare pussel själv
	\item Kan vi göra ett matematiskt påstående av dem, då kan vi göra sanningstabeller!
	\item Kan vi göra vissa enklare, genom att använda att och och eller inte alltid förändras?
	\item Kopplar även till matematiska påståenden, visa den enklaste saken
	\item Fokusera hela tiden på att strukturerat bevisa kluringar
	\item Extra uppgifter: Några svårare pussel + Bevis
\end{itemize}

\section{Mängdlära}
\begin{itemize}
	\item Vad är en mängd + exempel
	\item Vad kan vi göra för operationer på mängder?
	\item Påståenden om element
	\item Hur konjunktioner agerar på mängden av element som möter påståendet
	\item Extrauppgifter: Beräkna antalet element i dessa mängder. Skapa dessa mängder, Venn diagram
\end{itemize}

\section*{Binära tal}
En kvällsföreläsning
\begin{itemize}
	\item Här är binära tal
	\item Så här räknar vi med binära tal
	\item Användbara eftersom de är 0 och 1
	\item Binär sökning
	\item Kan vi beskriva algoritmen för summan?
	\item Extra: Tvåkomplements form
\end{itemize}

\section{Kretsar}
\begin{itemize}
	\item Vad är kretsar?
	\item Hur kan vi skriva upp kretsar?
	\item Formulera kretsar
	\item Skapa ny operatorer
	\item Skapa en 1 bit adderare
	\item Skapa en större adderare
	\item Skapa en som gör två olika saker beroende på en till input
	\item Två komplements form
\end{itemize}














\newpage 
\newpage 

Att lära ut:
\begin{enumerate}
	\item Vad är ett påstående?
	\item När är det sant?
	\item Hur förhåller sig påstående till varrandra?
	\item Hur jobbar vi med deras sanningsvärden?
		\begin{enumerate}
			\item Sanningstabeller
			\item Kretsar
		\end{enumerate}
	\item Logik pussel
	\item Binära tal
	\item Att skapa en miniräknare
\end{enumerate}


Funderar på att skippa mängdläran, och gå med sanningstabeller istället?

\newpage 
\section{Struktur}
\subsection{Logiska verktyg}
\begin{enumerate}
	\item Påståenden och förhållanden:
		\begin{itemize}
			\item Vad är och är inte ett påstående?
			\item Negation
			\item Och, eller, xeller
			\item Implikation, ekvivalens
			\item Kvantorer
		\end{itemize}
	\item Sanningstabeller och Pussel
		\begin{itemize}
			\item 1 och 0
			\item Sanningstabeller
			\item Varulvar-Pussel
			\item Kluringar
		\end{itemize}
	\item Mängdlära
		\begin{itemize}
			\item Vad är en mängd?
			\item Olika mängder
			\item Operatorer på mängder
			\item Mängdlära och logik
		\end{itemize}
	\item Bevis
		\begin{itemize}
			\item Vad ett bevis är
			\item Utesluta alla alternativ
			\item Typer av bevis
		\end{itemize}
\end{enumerate}

\subsection{Datorer}
\begin{enumerate}
	\item Påståenden och förhållanden:
		\begin{itemize}
			\item Vad är och är inte ett påstående?
			\item Negation
			\item Och, eller, xeller
			\item Implikation, ekvivalens
			\item Kvantorer
		\end{itemize}
	\item Sanningstabeller och Pussel
		\begin{itemize}
			\item 1 och 0
			\item Sanningstabeller
			\item Varulvar-Pussel
			\item Logik i bevis?
		\end{itemize}
	\item Kretsar och beslut
		\begin{itemize}
			\item Vad är kretsar?
			\item Hur kan vi skriva upp kretsar?
			\item Formulera kretsar
			\item Skapa ny operatorer
		\end{itemize}
	\item Binära tal
		\begin{itemize}
			\item Vad är binära tal
			\item Hur kan vi räkna med binära tal?
			\item Kan vi använda kretsar?
			\item Bygga en miniräknare
			\item Bygga vidare
		\end{itemize}
\end{enumerate}


\section{Bevis}
\begin{itemize}
	\item Satser och definitioner
	\item Direkta bevis
	\item Indirekta bevis
	\item Motsägelse bevis
	\item Alltid med fokus på logiken bakom, och lite roligare exempel
	\item Visa att detta stämmer
	\item Göra matematiska bevis, och skriv ned dem
	\item Extra: Induktionsbevis
\end{itemize}
